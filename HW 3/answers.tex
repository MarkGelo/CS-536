% https://github.com/jdavis/latex-homework-template
% https://oeis.org/wiki/List_of_LaTeX_mathematical_symbols

\documentclass{article}

\usepackage{fancyhdr}
\usepackage{extramarks}
\usepackage{amsmath}
\usepackage{amsthm}
\usepackage{amssymb}
\usepackage{amsfonts} % for "\mathbb" macro
\newcommand{\N}{\mathbb{N}}
\newcommand{\Z}{\mathbb{Z}}
\newcommand{\Q}{\mathbb{Q}}
\newcommand{\R}{\mathbb{R}}
\newcommand{\C}{\mathbb{C}}
\usepackage{tikz}
\usepackage[plain]{algorithm}
\usepackage{algpseudocode}
\usepackage{enumitem}
\usetikzlibrary{automata,positioning}
\newcommand{\indentitem}{\setlength\itemindent{25pt}}
%
% Basic Document Settings
%

\topmargin=-0.45in
\evensidemargin=0in
\oddsidemargin=0in
\textwidth=6.5in
\textheight=9.0in
\headsep=0.25in

\linespread{1.1}

\pagestyle{fancy}
\lhead{\hmwkAuthorName}
\chead{\hmwkClassID\ \hmwkTitle}
\rhead{} % blank, to remove the latest section in page
\cfoot{\thepage}

\renewcommand\headrulewidth{0.4pt}
\renewcommand\footrulewidth{0.4pt}

\setlength\parindent{0pt}

\setcounter{secnumdepth}{0}

%
% Homework Details
%   - Title
%   - Due date
%   - Class
%   - Section/Time
%   - Instructor
%   - Author
%

\newcommand{\hmwkTitle}{Homework\ \#3}
\newcommand{\hmwkDueDate}{February 21, 2022}
\newcommand{\hmwkClass}{Science of Programming}
\newcommand{\hmwkClassID}{CS 536}
\newcommand{\hmwkClassInstructor}{Stefan Muller}
\newcommand{\hmwkAuthorName}{\textbf{Mark Gameng}}

%
% Title Page
%

\author{\hmwkAuthorName}
\date{}

%
% Various Helper Commands
%

% Useful for algorithms
\newcommand{\alg}[1]{\textsc{\bfseries \footnotesize #1}}

% For derivatives
\newcommand{\deriv}[1]{\frac{\mathrm{d}}{\mathrm{d}x} (#1)}

% For partial derivatives
\newcommand{\pderiv}[2]{\frac{\partial}{\partial #1} (#2)}

% Integral dx
\newcommand{\dx}{\mathrm{d}x}

% Alias for the Solution section header
\newcommand{\solution}{\textbf{\large Solution}}

% Probability commands: Expectation, Variance, Covariance, Bias
\newcommand{\E}{\mathrm{E}}
\newcommand{\Var}{\mathrm{Var}}
\newcommand{\Cov}{\mathrm{Cov}}
\newcommand{\Bias}{\mathrm{Bias}}

\newcommand{\answer}{\item[]} %new code

\begin{document}
	
	%\maketitle
	
	\section{Task 1.1}
	
		\begin{enumerate}[label={(\alph*)}]
			
			\answer $i = 0; while(i < size(a))\{x = x + a[i]; i = i + 1 \}$
			
		\end{enumerate}
	
	\section{Task 2.1}
		\begin{enumerate}[label = {(\alph*)}]
			
			\item 
			$\begin{aligned}[t]
				& <s, {n = 5}> \\
				&\rightarrow \quad <b; s, {n = 5}> \\
				&\rightarrow \quad <n = 3 * n + 1; s, {n = 5}> \\
				&\rightarrow \quad <s, {n = 16}> \\
				&\rightarrow \quad <b; s, {n = 16}> \\
				&\rightarrow \quad <n = n / 2; s, {n = 16}> \\
				&\rightarrow \quad <s, {n = 8}> \\
				&\rightarrow \quad <b; s, {n = 8}> \\
				&\rightarrow \quad <n = n / 2; s, {n = 8}> \\
				&\rightarrow \quad <s, {n = 4}> \\
				&\rightarrow \quad <b; s, {n = 4}> \\
				&\rightarrow \quad <n = n / 2; s, {n = 4}> \\
				&\rightarrow \quad <s, {n = 2}> \\
				&\rightarrow \quad <b; s, {n = 2}> \\
				&\rightarrow \quad <n = n / 2;s, {n = 2}> \\
				&\rightarrow \quad <s, {n = 1}> \\
				&\rightarrow \quad <skip, {n = 1}> \\
			\end{aligned}$
			
			\item $M(s, \sigma) = \{\{n = 1\}\}$
			
		\end{enumerate}
	
	\section{Task 2.2}
		\begin{enumerate}[label = {(\alph*)}]
			
			\item $M(s, \sigma) = \{\bot_{d} \}$
			
			\item $M(s, \sigma) = \{ \{n = 0\}\}$
			
			\item $M(s, \sigma) = \{ \{n = -1\}\}$
			
			\item $M(s, \sigma) = M(y := a[x], \sigma) = \{ \{x = 1, a = [0, 3, 2, 1], y = 3 \}\}$
			
			\item $M(s, \sigma) = M(z := 0, \sigma) = \{ \{x = -1, a = [0, 3, 2, 1], z = 0\}\}$
			
			\item $M(s, \sigma) = M(y := a[x], \{x = 5, a = [0, 3, 2, 1], z = 0\}) = \{\bot_{e}\}$
			
		\end{enumerate}
	
	\pagebreak
	
	\section{Task 3.1}
		\begin{enumerate}[label = {(\alph*)}]
			
			\item \textbf{Unsatisfied}. Post-condition is false, as $i$ becomes 0. 
			
			\item \textbf{Satisfied}. Pre-condition and post-condition passes. $i = 0 \land i \geq 0$ and $x = 6 \land x \geq 1$.
			
			\item \textbf{Satisfied}. Same reason as above. Program doesn't terminate and errors.
			
			\item \textbf{Satisfied}. Pre-condition is false thus, the triple is satisfied.
			
			\item \textbf{Satisfied}. Same reason as above.
			
			\item \textbf{Unsatisfied}. Post-condition is false, as $x = 0$.
			
			\item \textbf{Unsatisfied}. Post-condition is false, since $i$ is updated throughout the program. At the end, $i = 0$, $x = 6$, and thus, $x \neq 0!$ or $x \neq 1$.
			
			\item \textbf{Satisfied}. Fixes the previous problem by saving $i$ in another variable, k. Precondition and post-condition passes. At the end, $i = 0 \land x = 6 \land k = 3$ and thus, $x = k!$.
			
		\end{enumerate}
	
	\section{Task 3.2}
		\begin{enumerate}[label = {(\alph*)}]
			
			\item Valid. If $y = 0$ then causes an error which is fine for partial correctness. For all other states, they terminate in a state satisfying the post-condition.
			
			\item Not valid since with $y = 0$ leads to runtime error, $x / 0$. Fixed: $[x \geq 0 \land y > 0] z:= x / y [z \geq 0]$
			
			\item Valid
			
			\item Not valid, since we don't know the contents of array, $a$. They can be negative which makes the post condition false. So, I changed the program and to make it use the absolute value instead. Fixed: $\{i \geq 0 \land i < |a|\} x := |a[i]| \{x \geq 0\}$
			
		\end{enumerate}
	
	\section{Task 3.3}
		\begin{enumerate}[label = {(\alph*)}]
			
			\answer Since $r := r * (-2)$, we need to make n be even so that it ends on a positive and passes the post-condition.
			$[r = 1 \land \exists k \in \Z . n = 2 * k] m := n;$ while $n \neq 0 \{r := r * (-2); n := n - 1\}[r = 2^m]$
		
		\end{enumerate}
	
	\section{Task 4.1}
		\begin{itemize}
			
			\answer I spent about 3 hours on this.
			
		\end{itemize}
	
\end{document}
